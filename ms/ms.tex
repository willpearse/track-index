\documentclass[12pt]{report}
\usepackage[utf8]{inputenc}
\usepackage{times}
\renewcommand\thesection{\arabic{section}}
\usepackage[parfill]{parskip}
\usepackage{subcaption}
\usepackage{graphicx}
\usepackage[left=0.75in,right=0.75in,top=0.75in,bottom=0.75in]{geometry}
\usepackage{multirow}
\usepackage{hhline}
\usepackage{nth}
\usepackage{bm}
\usepackage{siunitx}
\usepackage[labelfont=bf]{caption}
\usepackage[doublespacing]{setspace}
\usepackage{filecontents}
\usepackage{lineno} \linenumbers
\usepackage{amsmath}

%%%%%%%%%%%%%%%%%%%%%%%%%%%%%%%%%%%
%Bibliography%%%%%%%%%%%%%%%%%%%%%%
%%%%%%%%%%%%%%%%%%%%%%%%%%%%%%%%%%%
\usepackage[style=numeric-comp,sorting=none,uniquelist=minyear,uniquename=false,maxcitenames=2,maxbibnames=5,doi=false,url=false,isbn=false,defernumbers=true,giveninits=true]{biblatex}
%Italicise et al.
\renewbibmacro*{name:andothers}{
% Based on name:andothers from biblatex.def
\ifboolexpr{ test{\ifnumequal{\value{listcount}}{\value{liststop}}} and test \ifmorenames } {\ifnumgreater{\value{liststop}}{1} {\finalandcomma}
{}%
\andothersdelim\bibstring[\emph]{andothers}} {}}
%Make 'and' an ampersand
\renewcommand*{\finalnamedelim}{%
\ifnumgreater{\value{liststop}}{2}{\finalandcomma}{}%
\addspace\&\space}
%Remove 'in:'
\renewbibmacro{in:}{}
%Remove pp. on pages numbers
\DeclareFieldFormat[article]{pages}{#1}%
%Remove unwanted entries
\AtEveryBibitem{%
  \clearfield{day}%
  \clearfield{month}%
  \clearfield{endday}%
  \clearfield{endmonth}%
  \clearfield{url}%
  \clearfield{URL}%
  \clearfield{eprint}%
}
%Remove spaces between initials
%Remove double quotes around titles
\DeclareFieldFormat*{citetitle}{#1}
\DeclareFieldFormat*{title}{#1}
\addbibresource{/home/will/Papers/Library.bib}
\begin{document}
\section*{Title page}
\textbf{Article title}: Placing species' climate envelope within its
historic context reveals remarkable idiosyncrasy of distributional
change

\textbf{Authors:} William D.\ Pearse$^{1,*}$ \& T.\ Jonathan
Davies$^{2,*}$

$^1$ Department of Biology \& Ecology Center, Utah State University,
5305 Old Main Hill, Logan UT, 84322

$^2$ Department of Biological Sciences, Universit\'{e} du Qu\'{e}bec
\`{a} Montr\'{e}al, Montr\'{e}al, Qu\'{e}bec, Canada

$^*$ To whom correspondence should be addressed:
\url{will.pearse@usu.edu} and \url{j.davies@ubc.ca}

\textbf{Word-count}: XXX (excluding figure legends)

\clearpage
\textbf{Abstract. Here we
  show ... Abstract.}

% Species have climatic tolerances that are conserved
% * Caused by deep-time evolutionary processes we can detect
% * Tested by local-scale ecological assembly
% * Shown by niche-models
Species' distributions are determined by interacting factors such as
traits, climate, land-use, biotic interactions, and dispersal
limitation \supercite{Pulliam2000,Ricklefs2004,Vellend2010}. Unpicking
the relative contributions and mechanisms of these processes is
challenging, but critical if we are to make predictions about
biodiversity as these factors change simultaneously in the
Anthropocene \supercite{Heller2009,Oliver2014,Urban2015}. While
macroevolutionary biologists frequently argue that species'
distributions tend to shown phylogenetic conservatism, the tendency
for closely-related species to have similar niches
\supercite{Wiens2004,Wiens2010}. Likewise, ecologists have shown that
species' distributions are strongly driven by climate, either through
correlative studies \supercite{xxx} or experimental studies
\supercite{xxx}. While environment and climate are clearly not the
only determinants of species' distributions, decades of research into
niche distribution models have upheld the view that they are important
\supercite{xxx}.

% Species are responding to climate change
% * Species' present-day distribution still show phylogenetic signal
% * Species show local-scale assembly changes in manipulations and long-term studies
% * Ecological niche modelers show that distributions are changing
Species' distributions are undoubtedly responding to climate
\supercite{Parmesan2003,Parmesan2006} and land-use
\supercite{Newbold2015} changes. Species' contemporary distributions
still, however, exhibit phylogenetic signal \supercite{xxx}, implying
that these changes are continuations of past conditions. Equally,
experimental studies \supercite{xxx}, and the ever-increasing number
of observational studies of ongoing climate change \supercite{xxx},
have shown that species' distributions and local-scale ecological
assembly respond to climate. Niche modelers have shown that their
models have some predictive power over species' distributions given
past relationships with climate \supercite{xxx}. That there is broad
consensus that ranges are changing does not mean there have not been
debates over the precise implications and mechanisms of such
changes. For example, the implications of range change for local-scale
biodiversity is unclear
\supercite{Vellend2013,Dornelas2014,Gonzalez2016}, we do not know
whether species' environmental \emph{limits} or \emph{central
  tendencies} are most-strongly conserved
\supercite{deCasas2017,Zanne2018}, and there is a relative paucity of
historic data with which to create historic
baselines\supercite{Newbold2015,Meineke2018}.

% Yet we do not know the extent to which species are tracking their niches
% * We know there is idiosyncracy and that niche models do not always work well
% * Could this be because we have yet to define the null model by which they should be assembling?
% * Could this be because of a breakdown in the climate vectors of species? That heterogeneity is causing problems for species to track their climate?
% * Here we present a framework that accounts for this.
This consensus of species' past and present ranges broadly tracking
the same underlying climate optima is complicated by paleo-ecological
studies that have revealed greater variability of species' historic
ranges in previous centuries and millennia than previously thought
\supercite{Williams2007,Veloz2012,Maguire2015}. Critically, while
species' present-day distributions have changed, and they are broadly
similar to past distributions, we have yet to establish a historic
benchmark by which to assess the magnitude of that tracking. Many
species' distributions span continents, while others spans kilometers,
and the velocity of climate change is not constant across those scales
\supercite{xxx}. As climate continues to change, the underlying
associations between axes of climate have changed \supercite{xxx}: if
species are, as they must be \supercite{xxx}, ultimately limited by a
single environmental axis, what are the consequences of this change
for species' distributions? We argue there is a need for a synthetic
measure of the extent to which species have tracked their niche
through time, given the changes to which they have been exposed.

\subsection*{Results and discussion}
\textbf{A comparative index of species' niche tracking.} The efficacy
of niche models to predict species' distributions is, in part, a
measure of the extent to which species are tracking climate. Here we
present a simple metric of species' degree of track, which we define
as:

\begin{equation}
  track_\tau = \frac{present_\tau - past_\tau}{projected_\tau - past_\tau}
\end{equation}

Where $current_\tau$ is a given quantile ($\tau$) of a species'
climatic range in the present, $past_\tau$ the same quantile of a
species' range in the present, and $projected_\tau$ the same quantile
of a species' present-day range under past climatic conditions. Thus a
$track$ of 0 would indicate a species that has tracked changing
climate perfectly, while a $track$ of 1 would be consistent with a
species that had not moved whatsoever to track changing environmental
conditions. Values between 0 and 1 suggest moderate tracking, while
values greater than 1 or less than 0 represent a change in climate
relationship greater than if the species had not moved at
all. Uncertainty in either species' past or present ranges can be
accounted for through a bootstrap procedure, whereby the past and/or
present distributions are re-sampled (with replacement) to match the
sampling of the other time-period, and the standard deviation of the
index across those samples reported.

Our approach has two main innovations: (1) a focus on the whole of a
species' distribution and on (2) expected change. (1) We suggest that
something of a disconnect is developing among related fields of
ecology: species' range modelers typically focus on species'
limits\supercite{xxx} (\emph{i.e.}, the upper and lower quantiles)
while community ecologists focus on central tendencies\supercite{xxx}
(\emph{i.e.}, the median). While there is some evidence plants' global
distributions are more strongly driven by limits
\supercite{Zanne2018}, we see no strong \emph{a priori} reason that
species can, or must, track all quantiles equally\supercite{xxx}. Thus
our approach is to check all quantiles of a species' distribution. (2)
The denominator of our index re-scales species' change around an
expectation under distributional stasis. We suggest that species'
changes should be measured on the basis of the magnitude of change to
which they are exposed. This change reflects the magnitude of
ecological and evolutionary selection, and while we suggest there are
many ways of addressing this, we suggest ours is the simplest and so
perhaps a fair starting point.

\textbf{Niche tracking across the tree of life throughout the past
  sixty years.} To assess the extent to which species have tracked
climate, we used the most comprehensively sampled time-series data for
species' distributions (GBIF\supercite{xxx}) and climate (UEA's
CRU\supercite{xxx}). We focus on only five quantiles within the data:
the \nth{5}, \nth{25}, \nth{50} (median), \nth{75}, and \nth{95}. In
the methods we give more detail of our reproducible analysis pipeline,
and in the supplementary materials we repeat our analyses with more,
and less, stringent data curation and cleaning. Here, however, we
present species' range data collected from 1955--2015 for XXX species
(XXX plants, XXX fungi, XXX mammals, XXX birds, XXX reptiles, XXX
amphibians, and XXX insects), each observation matched to one of nine
climate variables in the year in which they were sampled (mean,
minimum, maximum temperature, diurnal temperature range,
precipitation, wet-day frequency, vapor pressure, and cloud
cover). We present, in the methods and supplementary materials, a more
detailed analysis of how we accounted for potential and
well-known\supercite{xxx} biases in these data, and we emphasize that
our metric was designed explicitly to account for the known increase
in sampling effort through time in GBIF data. We also highlight that
our data support known linkages between species' climatic ranges and
phylogenetic history and traits, suggesting our data reveal true
underlying signal.

Figure \ref{violin} shows the empirical index values for our focal
taxa for mean temperature. Clear patterns emerge: a large number of
species are, to some extent, tracking climate through time (XXX\%
track at least one quantile, XXX\% track at least three, and XXX\%
track all five). Of these taxa, XXX are, on average, tracking the best
(XXX\% track at least one quantile) while XXX track the worst (XXX\%
track at least one quantile).
% To discuss: these percentages go up if we use the absolutised
% version of the index, but I'm not sure that's appropriate anymore.
Results for the other eight climate variables are comparable, and are
presented in the supplementary materials. Notably, however, many taxa
are not tracking climate through time, and many have experienced more
change than if their distributions had remained static (XXX\% have
`overshot' in at least three quantiles). These results are consistent
with null simulations we performed of species' ranges that included
some degree of climate tracking and empirically observed amounts of
sampling effort changes through time and space in the GBIF data.

\textbf{Idiosyncratic drivers of species' tracking.} To examine the
potential drivers of species' ranges and range tracking, we first
examined the phylogenetic signal of species' climate quantiles. We
focused on mammals\supercite{xxx}, birds\supercite{xxx},
amphibians\supercite{xxx}, reptiles\supercite{xxx}, and
plants\supercite{xxx}, since we were able to find broadly inclusive,
dated phylogenies for these taxa. We found statistically significant
evidence of phylogenetic signal across XXX of 135
taxon-climate-quantile combinations of species' past ranges (and XXX
in the present; see figure \ref{phylogeny}). This is consistent with
the known patterns of phylogenetic conservatism outlined above. For
our index, however, we found evidence of phylogenetic signal in only
XXX combinations (XXXlist themXXX). This confirms the known results
that species' present-day and past ranges are strongly influenced by
evolutionary, but suggests that the magnitude of change (relative to
expectations) may be driven by different drivers.

We additionally examined species' traits as potential drivers of
species' distributions and potential to track changing environment. We
focused on mammals\supercite{xxx}, birds\supercite{xxx}, and
plants\supercite{xxx}, since global, openly-released trait datasets
exist for these taxa. Again, we found statistically significant
associations between traits and past distributions of XXX of XXX
taxon-climate-quantile combinations (XXX in the present; see figure
\ref{traits}). While the traits we used are strongly associated with
species' climate responses\supercite{xxx} (hence their strong
predictive power of species' distributions, but not tracking index),
we emphasize that we do not examine traits related to dispersal
ability. This is due to a paucity of open-access dispersal traits
\supercite{xxx}, and we highlight that future work should address
whether our index is explained by such drivers.
% Another potential problem here is no trait*env interaction, which I
% sense is also important...
% WILL - re-do TEMP.INDEX ~ TEMP! THERMOPHILES!!!

\textbf{Climate space is not the same as tracking space.} Given our
index is poorly predicted by traditionally important variables, we now
consider whether index values are predictive of each other. It is well
known that global climate\supercite{xxx}, and so species' positions
across the various axes of climate space\supercite{xxx}, are not
independent, and so we set out to test whether species' tracking
indices are also independent. As we show in figure \ref{pca}, a
principal components analysis of species' present, past, and projected
climate distributions shows strong correlations across climate
variables. Yet the index does not: species' relative degree of
tracking is both of much higher dimension, and shows relatively less
correlation across climates axes. We argue, therefore, that while
species' positions in climate space show strong associations, species'
relative tracking of their position in climate space does not.

If true, this result would explain the lack of association between
species' traits and evolutionary history with their degree of climate
tracking. The ecological and evolutionary rules that determine
species' climate relationships are changing, and thus the factors that
once are strong predictors of past and present distributions are not
predictive of the differences between them. Yet there are alternative
explanations, of which we highlight three: (1) that we have missed
fundamentally important environmental axes, (2) biotic interactions,
and (3) that our index highlights random variation in
distribution. (1) We fully acknowledge that we ignore potentially
relevant environmental drivers such as geology (and so soil), habitat,
and all Anthropogenic factors other than climate. We are not alone in
this simplification\supercite{xxx}, but we detect strong evolutionary
pattern in the factors we do examine in the data underlying our
index. (2) Competition and facilitation among species are likely
strong drivers of species' ability to track climate\supercite{xxx}. We
acknowledge it is entirely possible that ignoring these factors is the
driver of our result and, if so, we urge for an acceleration of the
inclusion of such information into broad-scale analyses
\supercite{xxx}. (3) If species were perfectly tracking climate
(\emph{i.e.}, $present_\tau$ is equal to $past_\tau$ in the absence of
sampling error or some other stochastic process) then our index values
would essentially be random numbers. There are good empirical evidence
that this is not the case \supercite{xxx} but, as we show in the
supplementary materials, simulations suggest out values are
incompatible with this result.

\subsection*{Conclusion}
Our results are consistent with the processes underlying species'
responses to climate change being distinct from those that generated
their ranges in the past. That is not to say that we do not detect the
imprint of ecological and evolutionary history, or that those earlier
processes are of predictive power for species distributions, but
rather that to understand change we must consider new drivers. We do
not know what those drivers are, although we highlight that biotic
interactions may play a part. We also highlight that our data,
although comprehensive, are not taxonomically complete (we address
only XXX species) and are necessarily spatially ($0.5^\circ$ grid
cells) and temporally (yearly) coarse given the extent (global and 60
years) of our study. We are not the first to suggest that, for large
numbers of species, the extent of climate tracking is idiosyncratic
\supercite{xxx}. These results are, however, consistent with the
paleo-ecological record. During and after the last ice-age,
`no-analogue' communities generated previously-unseen species
assemblages \supercite{xxx}. Once previous mass extinction events were
underway, species' extinctions became unpredictable
\supercite{xxx}. Scientists have been warning for decades that we are
beginning a sixth mass extinction;
% Soule 1999 but find something earlier
perhaps we are seeing the end of its beginning.

\clearpage
% Figures

\begin{figure}[h!]
  \begin{center}
    \includegraphics[width=0.7\textwidth]{../figures/violin-tmp-50.pdf}
  \end{center}
  \caption{\doublespacing \textbf{Empirical distributions of track
      index for \nth{50} of mean temperature.} Need to add simulations
    to plot. Add phylo-pics?}
  \label{pca}
\end{figure}

\begin{figure}[h!]
  \begin{center}
    \includegraphics[width=0.7\textwidth]{../figures/pca-50.pdf}
  \end{center}
  \caption{\doublespacing \textbf{Principal components analyses of
      present and past climate, along with our index of climate
      tracking.} Loadings are correlated (eigenvalues shown at bottom
    of plot), but much weaker overall in index.}
  \label{pca}
\end{figure}

\begin{figure}[h!]
  \begin{center}
    \includegraphics[width=0.7\textwidth]{../figures/phylo-tmp-50.pdf}
  \end{center}
  \caption{\doublespacing \textbf{Phylogeny with ancestral states
      branches of past temperature (50th quantile) and tips of the
      temperature 50th index.} Scale between the two kinds of input
    data currently is slightly off, hence no scale bar, and obviously
    the tip colours are hideous, but I think no one would disagree
    there's signal in the underlying data!}
  \label{pca}
\end{figure}

% zoom in on one clade and show that the tips are a random subset (but the branches are not)
% Will wonders if we can put the correlation matrix to the side of an upward-facing tree
% have lines going outward from the tree to show mapping

\begin{figure}[h!]
  \begin{center}
  \end{center}
  \caption{\doublespacing \textbf{Trait correlation box.}
    XXXmissingXXX}
  \label{traits}
\end{figure}

\clearpage

\textbf{Acknowledgments} WDP was funded by NSF ABI-1759965, NSF
EF-1802605, and USDA Forest Service agreement 18-CS-11046000-041. TJD
was funded by \emph{Fonds de Recherche Nature et Technologies} grant
number 168004.

\textbf{Author Contributions} WDP and TJD contributed to all aspects
of the study.

\textbf{Author information} Reprints and permissions information is
available at \url{www.nature.com/reprints}. The authors declare no
competing financial interests. Correspondence and requests for
materials should be addressed to \url{will.pearse@usu.edu}.

\textbf{Competing interests} The authors declare no competing
financial interests

\clearpage

\section*{\Large Methods}

\printbibliography

\end{document}


%%% Local Variables:
%%% mode: latex
%%% TeX-master: t
%%% End:

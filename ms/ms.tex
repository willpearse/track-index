\documentclass[12pt]{report}
\usepackage[utf8]{inputenc}
\usepackage{times}
\renewcommand\thesection{\arabic{section}}
\usepackage[parfill]{parskip}
\usepackage{subcaption}
\usepackage{graphicx}
\usepackage[left=0.75in,right=0.75in,top=0.75in,bottom=0.75in]{geometry}
\usepackage{multirow}
\usepackage{hhline}
\usepackage{nth}
\usepackage{bm}
\usepackage{siunitx}
\usepackage[labelfont=bf]{caption}
\usepackage[doublespacing]{setspace}
\usepackage{filecontents}
\usepackage{lineno} \linenumbers
\usepackage{amsmath}
\usepackage{pdflscape}

%%%%%%%%%%%%%%%%%%%%%%%%%%%%%%%%%%%
%Bibliography%%%%%%%%%%%%%%%%%%%%%%
%%%%%%%%%%%%%%%%%%%%%%%%%%%%%%%%%%%
\usepackage[style=numeric-comp,sorting=none,uniquelist=minyear,uniquename=false,maxcitenames=2,maxbibnames=5,doi=false,url=false,isbn=false,defernumbers=true,giveninits=true]{biblatex}
%Italicise et al.
\renewbibmacro*{name:andothers}{
% Based on name:andothers from biblatex.def
\ifboolexpr{ test{\ifnumequal{\value{listcount}}{\value{liststop}}} and test \ifmorenames } {\ifnumgreater{\value{liststop}}{1} {\finalandcomma}
{}%
\andothersdelim\bibstring[\emph]{andothers}} {}}
%Make 'and' an ampersand
\renewcommand*{\finalnamedelim}{%
\ifnumgreater{\value{liststop}}{2}{\finalandcomma}{}%
\addspace\&\space}
%Remove 'in:'
\renewbibmacro{in:}{}
%Remove pp. on pages numbers
\DeclareFieldFormat[article]{pages}{#1}%
%Remove unwanted entries
\AtEveryBibitem{%
  \clearfield{day}%
  \clearfield{month}%
  \clearfield{endday}%
  \clearfield{endmonth}%
  \clearfield{url}%
  \clearfield{URL}%
  \clearfield{eprint}%
}
%Remove spaces between initials
%Remove double quotes around titles
\DeclareFieldFormat*{citetitle}{#1}
\DeclareFieldFormat*{title}{#1}
\addbibresource{/home/will/Papers/Library.bib}
\begin{document}

% To-do before bioRxiv:
% * p-values in lambda
% * change years in simulation
% * check taxa numbers in violin plot
% * add copyright information for phylopics
% * add sigma into the range simulation code
% * not a track index. But what is it?...
% * can you get more birds into the manuscript? They seem to be being dropped somewhere (perhaps at the pre-processing step?)
\section*{Title page}
\textbf{Article title}: The realised velocity of climate change
reveals remarkable idiosyncrasy of species' distributional shifts

\textbf{Authors:} William D.\ Pearse$^{1,*}$ \& T.\ Jonathan
Davies$^{2,*}$

$^1$ Department of Biology \& Ecology Center, Utah State University,
5305 Old Main Hill, Logan UT, 84322

$^2$ Department of Biological Sciences, University of British
Columbia, Vanouver, Canada

$^*$ To whom correspondence should be addressed:
\url{will.pearse@usu.edu} and \url{j.davies@ubc.ca}

\textbf{Word-count}: XXX (excluding figure legends)

\clearpage
\textbf{To date, our understanding of how species have shifted in
  response to recent climate warming has been based on a few studies
  with limited number of species. Here we present a comprehensive,
  global overview of species' distributional responses to changing
  climate across a broad variety of taxa (animals, plants, and
  fungi). We characterise species' responses using a metric that
  describes the realised velocity of climate change: how closely
  species' responses have tracked changing climate through time
  climate. In contrast to existing `climate velocity' metrics that have focused on space,
  we focus on species and their null expectation of change
  % The null expectation bit is new; I think this is a novelty, no?
  , and thus can examine drivers of inter-specific variation. Here we
  show that species are tracking climate on average, but not
  sufficiently to keep up with the pace of climate change. Further,
  species responses are highly idiosyncratic, and thus previous
  projections assuming uniform responses may be misleading. This is in
  stark contrast to species' present-day and historical niche
  dimensions, which show strong evidence of the imprint of
  evolutionary history and functional traits. Our analyses are a first
  step in exploring the vast wealth of empirical data on species'
  historic responses to recent climate change.}

The natural environment is changing rapidly. Over the last century,
habitats have become increasingly fragmented and
isolated\supercite{Haddad2015}, the climate has become warmer, and
extreme climate events more frequent\supercite{IPCC2014}. Biological
diversity---the species of plants and animals with which we cohabit
the Earth---has responded predictably. Present-day extinction rates
are estimated to be up to three orders of magnitude greater than background
rates \supercite{Pimm2014} and are projected to increase further over
the next several decades \supercite{MEA2005}. Current estimates
suggest that over one million species may be threatened with
extinction\supercite{IPBESland2019}, and that we have experienced a
50\% decline in animal diversity in the past 40 years
\supercite{McLellan2014}.

The main direct human-induced drivers that impact biodiversity now are
habitat loss \supercite{Newbold2015} and fragmentation
\supercite{Haddad2015}, but climate change is likely to become a
dominant driver in the next few
decades\supercite{Parmesan2003,Thuiller2007,Urban2015}. Projected
impacts of climate change on biodiversity have attracted much
attention, but the uncertainty around the magnitude of future
extinctions (\emph{e.g.}, 37\%\supercite{Thomas2004} vs.\
7.9\%\supercite{Urban2015} of species) means more work is needed.  In
the face of changing climate species must either shift in space, to
track favourable conditions, or time, for example, flowering and
breeding earlier, or migrating sooner. Evidence suggests they are
doing both \supercite{Parmesan2003,Parmesan2006,Menzel2006,Chen2011},
but not all species are responding equally. Some species are expanding
their distributions, sometimes invasively \supercite{Hellmann2008},
and some species are flowering later\supercite{Cook2012}; species may
also be simply becoming less predictable
\supercite{Pearse2017phenology}, perhaps as evolved responses to
environmental cues break down.  Simply quantifying species' mean
responses ignores the important variability in responses among species
\supercite{Wolkovich2012}. Developing robust predictions of ecological
responses to climate change thus requires data from diverse species,
ecosystems, and climates to account for this large inter-specific
variation.
% Better, I think, thank you. I think there was something about the
% previous phrasing that sounded a bit too much like we were knocking
% people's stats, rather than the concepts, if that makes sense?

The velocity of
temperature change---the rate a species would have to shift its range
to maintain constant temperature---has been calculated at 0.42 km/yr
\supercite{Loarie2009}. A meta-analysis of taxonomic groups estimated
the average pole-ward shift in species' distributions has been
approximately 1.69 km/yr \supercite{Chen2011},
suggesting that species might be capable of keeping pace with shifting
climate.
% You questioned the 1.69 figure - this text is from you! :D
% - I agree, though, that I'm not sure this estimate is fair. I looked
%   in the supplement and there's a remarkable lack of data going into
%   the estimate, to be honest with you... Is this sufficiently polite
%   to address that? They also don't address the file-drawer effect at all.
Yet this estimate is derived from just 21 studies with taxonomic bias
reflecting the state of the field at the time (361 birds, 9 mammals,
and no plants)
% I want to make sure we're nice about this point, because it's a
% meta-analysis so it's hardly their fault! Also one of my PhD
% supervisors is on this paper :p
, and critically not address the large variation in response among
species or within taxa. Here we leverage the vast volume of
distributional data stored in digital data
repositories\supercite{url_gbif} to provide the first global synthesis
of species distributional responses to climate change. By contrasting
historical niche dimensions with those of the present day, we can
characterise the realised biotic velocity of climate change, which
represents the actual velocity with which species track climate across
space.

We show that, on average, species' historical (pre-1980) ranges have
warmed significantly, and that species have generally shifted their
distributions to ameliorate this warming, but not sufficiently to keep
pace with climate change. Yet this general pattern Yet this general
pattern masks a remarkably degree of variability in species responses:
many species' ranges are now even warmer than if their distibutions
had remained static. Further, we find that, while species' niche
dimensions are strongly predicted by their evolutionary history and
functional traits, the same is not true of their degree of
tracking. Thus while species' distributions are reasonably
predictable, their changes in distributions are not.

\clearpage
\subsection*{Results and discussion}
\textbf{A comparative index of species' niche tracking.} We propose to
measure the extent to which a species' distribution is tracking
climate change by standardising its observed change by the change it
would have experienced had it not shifted its distribution. Formally,
we define this as:

\begin{equation}
  track_\tau = \frac{current_\tau - past_\tau}{projected_\tau - past_\tau}
\end{equation}

Where $current_\tau$ is a given quantile ($\tau$) of a species'
climatic distribution across its present-day range, $past_\tau$ the
same quantile of a species' range in the past, and $projected_\tau$
the same quantile of a species' present-day range under past climatic
conditions. This index should be calculated separately for
temperature, pollution, or whatever environmental variable is of
interest. Thus a $track$ of 0 would indicate a species that has
tracked changing climate perfectly, while a $track$ of 1 would be
consistent with a species that had not moved whatsoever to track
changing environmental conditions. Values between 0 and 1 suggest
moderate tracking, while values greater than 1 or less than 0
represent a change in climate relationship greater than if the species
had not moved at all. Uncertainty in either species' past or current
ranges can be accounted for through a bootstrap procedure, whereby the
past and/or current distributions are re-sampled (with replacement) to
match the sampling of the other time-period, and the standard
deviation of the index across those samples reported (see methods).

Our approach has two main innovations: its focus on (1) the whole of a
species' distribution and (2) expected change. (1) Species' range modelers typically focus on species'
limits\supercite{Sexton2009} (\emph{i.e.}, the upper and lower
quantiles) while community ecologists \supercite{Tilman2004} and
macro-evolutionary \supercite{Zanne2018} biologists focus on central
tendencies (\emph{i.e.}, the median). We see no strong
\emph{a priori} reason that species can, or must, track all quantiles
equally. Indeed some evidence suggests different constraints on species
warm range limts versus their cool range limits, with biotic interactions more influential in the former,
and climate determining the latter\supercite{Freeman2018}.
It is possible, therefore, that shifts at species' range edges may be disconnected
from changes towards the centre of their distribution.
% phrase better?
Our approach is to check all quantiles of a species' distribution. (2)
The denominator of our index re-scales species' change around an
expectation under distributional stasis. We suggest that species'
distributional responses should be measured on the basis of the
magnitude of change to which they are exposed. This change reflects
the magnitude of ecological and evolutionary selection, and while
there are many ways of addressing this, we suggest ours is perhaps the
simplest and most transparent, so perhaps represents a fair starting
point. We note that we could also rescale our index according to the
distance that would need to be travelled to maintain climate stasis
(\emph{sensu} \citeauthor{Loarie2009}\supercite{Loarie2009}), but
prefer to present here a dimensionless (and so unitless) index of
tracking, which is simpler to compare across taxa and between
biogeographic regions.
% You ask if we could make a map of this. We could, but I don't think
% we could by Tuesday...

\textbf{Niche tracking across the tree of life throughout the past
  sixty years.} To assess the extent to which species have tracked
climate, we used the most comprehensively sampled time-series data for
species' distributions (GBIF\supercite{url_gbif}) and climate (UEA's
CRU\supercite{Harris2014}). This represents over 320 Gb of occurrence
data, collected over the past century.
% Perhaps now you see why I keep having panic attacks about running
% everything through!...
We focus on five quantiles: the \nth{5},
\nth{25}, \nth{50} (median), \nth{75}, and \nth{95}. Here, we present species' range data collected from 1955--2015 for 10,700 species (4,879 plants,
1,517 fungi, 273 mammals, 433 birds, 147 reptiles and amphibians, and
3,451 insects), each observation matched to one of nine climate
variables in the year in which they were sampled (mean, minimum,
maximum temperature, diurnal temperature range, precipitation, 'rainy-day'
frequency, vapour pressure, and cloud cover). In the methods we
give more detail of our reproducible analysis pipeline, and in the
supplementary materials we repeat our analyses with different data
curation and cleaning choices. Our metric was designed to account for known biases in the GBIF data \supercite{Beck2014}, and
we present in the supplementary materials additional analyses that
address and account for such biases.

Figure \ref{violin} shows observed index values for mean temperature.
Almost all species, 99.5\%, are tracking some climate variable through time, and a large number of
species are, to some extent, tracking temperature through time (65\%
track at least one quantile, 43\% track at least two, and 4\% track
all five). The three temperature indices are the least-tracked of any
climate variable, while precipitation is the most-tracked (86\% of
species track at least one of its quantiles). Notably, however, many
taxa are not tracking climate through time, and many have experienced
more change than if their distributions had remained static (99.9\% of
species have `overshot' in at least three quantiles). As we show in
figure \ref{violin}, and explore further in the supplementary
materials, these results are consistent with our null simulations
where species' range change is a combination of essentially random
movement and a moderate degree of subsequent climate filtering.
% Jonathan is getting confused about the numbers
% 99% vs. XXX%; be clearer

\textbf{Idiosyncratic drivers of species' tracking.} To examine the
potential drivers of species' ranges and range tracking, we first
examined the phylogenetic signal of species' climate quantiles. We
focused on mammals\supercite{Faurby2015}, birds\supercite{Jetz2012},
amphibians \supercite{Jetz2018}, reptiles \supercite{Zheng2016}, and
plants\supercite{Smith2018}, since we were able to find broadly
inclusive, dated phylogenies for these taxa. We found statistically
significant evidence of phylogenetic signal across 158 of 180
taxon-climate-quantile combinations of species' past ranges (and 153
in the present; see figure \ref{phylogeny}). This is consistent with
the known patterns of phylogenetic conservatism outlined above. For
our index, however, we found evidence of phylogenetic signal in only
29 combinations. This confirms the known results that species'
current-day and past ranges are strongly influenced by evolutionary,
but suggests that the magnitude of change (relative to expectations)
may be driven by different drivers.
% Will: this is based on lambda *values* and so you need to update
% this with p-values...

We additionally examined species' traits as potential drivers of
species' distributions and potential to track changing environment. We
focused on mammals\supercite{Jones2009a,Wilman2014},
birds\supercite{Wilman2014}, and plants\supercite{Wright2004}, since
global, openly-released trait datasets exist for these taxa. Again, we
found statistically significant associations between traits and past
distributions of 193 of 240 taxon-climate-quantile combinations (203
in the present at $\alpha_{5\%}$; see figure \ref{traits}), but not
for between the traits and our tracking index (9 out of 240l
4\%). While the traits we used are strongly associated with species'
climate responses\supercite{Wright2004} (hence their strong predictive
power of species' distributions, but not tracking index), we emphasise
that we do not examine traits related to dispersal ability. This is
due to a paucity of open-access dispersal traits
\supercite{Gallagher2019}, and we highlight that future work should
address whether our index is explained by such drivers.
% Another potential problem here is no trait*env interaction, which I
% sense is also important...
% WILL - re-do TEMP.INDEX ~ TEMP! THERMOPHILES!!!

\textbf{Climate space is not the same as tracking space.} Given our
index is poorly predicted by traditionally important variables, we now
consider whether index values are predictive of each other. It is well
known that global climate\supercite{xxx}, and so species' positions
across the various axes of climate space\supercite{xxx}, are not
independent, and so we set out to test whether species' tracking
indices are also independent. As we show in figure \ref{pca}, a
principal components analysis of species' current, past, and projected
climate distributions shows strong correlations across climate
variables. Yet the index does not: species' relative degree of
tracking is both of much higher dimension, and shows relatively less
correlation across climates axes. We argue, therefore, that while
species' positions in climate space show strong associations, species'
relative tracking of their position in climate space does not.

If true, this result would explain the lack of association between
species' traits and evolutionary history with their degree of climate
tracking. The ecological and evolutionary rules that determine
species' climate relationships are changing, and thus the factors that
once are strong predictors of past and current distributions are not
predictive of the differences between them. A likely 


Yet there are alternative
explanations, of which we highlight three: (1) that we have missed
fundamentally important environmental axes, (2) biotic interactions,
and (3) that our index highlights random variation in
distribution. (1) We fully acknowledge that we ignore potentially
relevant environmental drivers such as geology (and so soil), habitat,
and all Anthropogenic factors other than climate. We are not alone in
this simplification\supercite{xxx}, but we detect strong evolutionary
pattern in the factors we do examine in the data underlying our
index. (2) Competition and facilitation among species are likely
strong drivers of species' ability to track climate\supercite{xxx}. We
acknowledge it is entirely possible that ignoring these factors is the
driver of our result and, if so, we urge for an acceleration of the
inclusion of such information into broad-scale analyses
\supercite{xxx}. (3) If species were perfectly tracking climate
(\emph{i.e.}, $current_\tau$ is equal to $past_\tau$ in the absence of
sampling error or some other stochastic process) then our index values
would essentially be random numbers. There are good empirical evidence
that this is not the case \supercite{xxx} but, as we show in the
supplementary materials, simulations suggest out values are
incompatible with this result.

\subsection*{Conclusion}
Our results are consistent with the processes underlying species'
responses to climate change being distinct from those that generated
their ranges in the past. That is not to say that we do not detect the
imprint of ecological and evolutionary history, or that those earlier
processes are of predictive power for species distributions, but
rather that to understand change we must consider new drivers. We do
not know what those drivers are, although we highlight that biotic
interactions may play a part. We also highlight that our data,
although comprehensive, are not taxonomically complete (we address
only XXX species) and are necessarily spatially ($0.5^\circ$ grid
cells) and temporally (yearly) coarse given the extent (global and 60
years) of our study. We are not the first to suggest that, for large
numbers of species, the extent of climate tracking is idiosyncratic
\supercite{xxx}. These results are, however, consistent with the
paleo-ecological record. During and after the last ice-age,
`no-analogue' communities generated previously-unseen species
assemblages \supercite{xxx}. Once previous mass extinction events were
underway, species' extinctions became unpredictable
\supercite{xxx}. Scientists have been warning for decades that we are
beginning a sixth mass extinction;
% Soule 1999 but find something earlier
perhaps we are seeing the end of its beginning.

This consensus of species' past and current ranges broadly tracking
the same underlying climate optima is complicated by paleo-ecological
studies that have revealed greater variability of species' historic
ranges in previous centuries and millennia than previously thought
\supercite{Williams2007,Veloz2012,Maguire2015}.

\clearpage
% Figures
\clearpage
\begin{figure}[h!]
  \begin{center}
    \includegraphics[width=\textwidth]{../figures/violin-tmp-0-5.pdf}
  \end{center}
  \caption{\doublespacing \textbf{Most species are tracking global
      temperature, but there is pronounced variation among species and
      clades.} The vertical axis shows the \nth{50} quantile of our
    tracking index of annual mean temperature (boostrapped to account
    for variation in sampling; see methods). On the left, our seven
    major taxonomic groups plotted with the median and \nth{25} and
    \nth{50} quantiles of their distributions highlighted. At the
    far-right of the plot, output from our simulations of species that
    are moving but either imperfectly tracking, or not tracking at
    all, are shown. In the main text, we highlight that the majority
    of species are tracking global climate, but there is profound
    variation among species and clades. Fungi, insects, and reptiles
    are, broadly, tracking the \nth{50} quantile of mean temperature
    well, while other taxa (\emph{e.g.}, mammals and plants) are not.}
  \label{violin}
\end{figure}

\clearpage
\begin{landscape}
\begin{figure}[h!]
  \begin{center}
    \includegraphics[width=1.4\textwidth]{../figures/phylo-tmp-50.pdf}
  \end{center}
  \caption{\doublespacing \textbf{Despite strong phylogenetic signal
      in species' static distributions, species' changes in
      distributions are not phylogenetically patterned.} This is a
    single sample from the posterior distribution of plant phylogenies
    used for our analysis\supercite{Smith2018}. The two oldest
    branches have been shortened for clarity of presentation
    (emphasised with dashed lines). Its internal branches are
    ancestral state reconstructions, assuming Brownian motion, of
    species' median mean-temperature in the past. Its tips are
    coloured according to the track index of the same underlying data
    (compare with figure \ref{violin}). Both are coloured according to
    the legend at the bottom of the figure. While this is but a single
    plot (see also figure \ref{correls} and the supplementary
    materials), this general pattern of strong phylogenetic signal of
    climate, but not of degree of tracking, is true of almost all of
    our underlying data (see main text).}
  \label{pca}
\end{figure}
\end{landscape}

\clearpage
\begin{figure}[h!]
  \begin{center}
    \begin{subfigure}{.45\textwidth}
      \includegraphics[width=\textwidth]{../figures/traits-50.pdf}
      \caption{Trait correlations}
    \end{subfigure}
    \begin{subfigure}{.45\textwidth}
      \includegraphics[width=\textwidth]{../figures/physig-5.pdf}
      \caption{Phylogenetic signal}
    \end{subfigure}
  \end{center}
  \caption{\doublespacing \textbf{Species' static climate variables
      show strong trait correlations and phylogenetic signal, but
      their relative degree of tracking does not.} In (a), we show the
    correlation between the median of each climate variable (rows)
    with each of our traits (columns); colours are given at the right
    of the figure. In each cell, the outer, larger box depicts the
    correlation between species' present-day climate and the trait,
    while the smaller, inner box the correlation for species' relative
    tracking (whose size is proportional to intensity of
    correlation). The 3 (of 54) track-index correlations that are
    statistically significant (at $\alpha_{5\%}$) are highlighted with
    crosses. In (b), we show the phylogenetic signal (Pagel's
    $\lambda$\supercite{Pagel1999}) of the median of each taxonomic
    group's (columns) median species' climate variables and relative
    tracking indices (rows). As with (a), in each cell the outer,
    larger box depicts the correlation between species' present-day
    climate and the trait, while the smaller, inner box the
    correlation for species' relative tracking (whose size is
    proportional to phylogenetic signal strength). The single
    track-index metric for which phylogenetic signal was stronger than
    the underlying data is highlighted with a cross.}
  \label{traits}
\end{figure}

\clearpage
\begin{figure}[h!]
  \begin{center}
    \begin{subfigure}{.45\textwidth}
      \includegraphics[width=\textwidth]{../figures/pca-05-past.pdf}
      \caption{Past climate space (1955--1980)}
    \end{subfigure}
    \begin{subfigure}{.45\textwidth}
      \includegraphics[width=\textwidth]{../figures/pca-05-present.pdf}
      \caption{Present climate space (1990--2015)}
    \end{subfigure}
    \begin{subfigure}{.45\textwidth}
      \includegraphics[width=\textwidth]{../figures/pca-05-index.pdf}
      \caption{Tracking climate space}
    \end{subfigure}
  \end{center}
  % Note: bad filenames I know, but these are the 50th (i.e., median) entries
  \caption{\doublespacing \textbf{Species' movement through climate
      space differs from their current positions within it.} Each of
    our figures is a principal component analysis (PCA) biplots; in
    (a) of species' median past climate values, in (b) median current
    climate, and in (c) the median track-index of all climate
    variables. A PCA biplot shows correlations among the underlying
    data, with each axis explaining some amount of variation in the
    underlying data (labelled on the axes themselves). In (a) and (b),
    species' climate variables are correlated with one-another, and
    the space reflects global patterns in climate. In (c), this
    pattern has broken down: while some associations among climate
    axes still remain (notably temperature), the general pattern is
    one of independent shift through climate space.}
  \label{pca}
\end{figure}

\clearpage

\textbf{Acknowledgements} WDP was funded by NSF ABI-1759965, NSF
EF-1802605, and USDA Forest Service agreement 18-CS-11046000-041. TJD
was funded by \emph{Fonds de Recherche Nature et Technologies} grant
number 168004.

\textbf{Author Contributions} WDP and TJD contributed to all aspects
of the study.

\textbf{Author information} Reprints and permissions information is
available at \url{www.nature.com/reprints}. The authors declare no
competing financial interests. Correspondence and requests for
materials should be addressed to \url{will.pearse@usu.edu}.

\textbf{Competing interests} The authors declare no competing
financial interests

\clearpage

\section*{\Large Methods}
\section*{\Large Methods}
\renewcommand{\figurename}{Extended Data Figure}
\renewcommand{\tablename}{Extended Data Table}
  
All analyses were conducted in \emph{R} version \texttt{3.6.1}
\autocite{R2018}, and all names in \emph{italics} refer to \emph{R}
packages. Code to reproduce all analyses in their entirety is
available in the online supplementary materials.

\subsection*{Data collation}

Species' distribution data were downloaded from the Global
Biodiversity Information Facility (GBIF)
\supercite{url_gbif,url_gbif_occurrences}. We downloaded only data
highlighted by GBIF as not having any spatial issues, meaning they
contained no obvious location errors (\emph{e.g.}, latitude/longitude
swaps or rounded coordinates). These occurrence data include records
that were flagged as having potential issues and corrected by GBIF;
our results were qualitatively identical if we excluded these records
and so we include them here for completeness. We used `human
observation' occurrences (\emph{e.g.}, records from surveys), and not
vouchered specimens (\emph{e.g.}, pressed specimens in a herbarium)
because preliminary analysis revealed qualitatively identical results
across the two methods, but with many fewer observations vouchered
observations. We only worked with species that had at least 1000
records on GBIF within our focal time periods (1955--1980 and
1985--2015). This pragmatic threshold limits us to better-studied
species, but our results are robust to increases (\emph{e.g.}, 10,000
observations) in this threshold. We only made use of data for which
species identity was known, and we ignored variation among
sub-species. The GBIF taxonomy is, itself, to some extent cleaned, and
so we retained in our analysis any species within GBIF that was not
present in another datasets (see
\citeauthor{Cornwell2019}\supercite{Cornwell2019} for a discussion of
the GBIF taxonomy for plants).

Climate data were downloaded from
\citeauthor{Harris2014}\supercite{Harris2014}'s nine global
time-series. We split the data into yearly (averaged) 0.5$^\circ$
cells, from which we estimated each of our nine climate variables for
each species observation using
\emph{raster}\supercite{Hijmans2019}. Plant trait data were taken from
\citeauthor{Wright2004}\supercite{Wright2004} (leaf lifespan, leaf
mass-per-unit-area, leaf Nitrogen mass, and leaf photosynthetic
capacity), mammal body mass data from PanTHERIA\supercite{Jones2009a}
and EltonTraits \supercite{Wilman2014}, and bird body mass data from
EltonTraits \supercite{Wilman2014}. We were unable to find sufficient
coverage of the other taxa in this study in open-access trait
databases to facilitate further analysis. We used existing global
mammal\supercite{Faurby2015}, bird\supercite{Jetz2012}, amphibian
\supercite{Jetz2018}, plant (`ALLOTB' phylogeny
\supercite{Smith2018}), and reptiles\supercite{Zheng2016} phylogenies,
and in cases where posterior distributions of trees were available we
used a single draw from that distribution.

\subsection*{Calculating and testing our track index}

The sampling of species through time and across space is known to be
uneven across the data in GBIF \supercite{Beck2013,Beck2014}. To
ensure our method was statistically robust to such changes, we
employed a bootstrap procedure during the calculation of
$track_\tau$. To do this we randomly re-sampled, with replacement, the
occurrences making up $current_\tau$ and $projected_\tau$ to be of the
same number as $past_\tau$, and vice-versa for $past_\tau$ and
$current_\tau$. We repeated this process 999 times, and calculated a
$track_\tau$ value for each re-sample, generating 999 $track_\tau$
values whose medians we report here.

To assess the performance of our (bootstrapped) track index, we
simulated species' whose ranges tracked climate to varying extents
through time. Using the climate data from above in the years 1962 and
2002 (the mid-points of the ranges of our data), we simulated species
with varying maximum possible range sizes (1x1, 2x2, 5x5, 10x10,
20x20, or 50x50 grid-cell extents) and varying possible latitudinal
shifts in range through time ($\pm$1, 2, 5, or 10 grid-cells). We also
varied two other parameters: a species' degree of environmental
tracking ($\alpha$; taking a value of 0, 0.5, or 1) and overall
fraction of chance occupancy ($\sigma$; taking a value of 0.5, .75, or
1). Together, these two parameters define a species' probability of
being present within a cell within its potential range:
$(1-\alpha) \times sigma + \alpha \times \mathcal{N}$.  $\mathcal{N}$
is defined as a scaled Normal probability density function with a mean
equal to the median of the species' distribution in 1962 and a
variance of 1. Thus a species with an $\alpha$ of 1 is present only in
cells within its range that resemble its climatic centre, and a
species with an $\alpha$ of 0 is present in proportion $\sigma$ of the
cells within its potential range. We simulated all possible
combinations of parameters 20 times, providing a full range of
potential range shifts.
% This needs more work!

\subsection*{Trait and phylogenetic analyses}

We calculated the Pearson's correlation between all estimated climatic
niche variables ($current_\tau$, $past_\tau$, $projected_\tau$,
$track_\tau$ and the bootstrapped $track_\tau$). While we show only a
handful of these correlations in figure \ref{traits}, we provide all
correlations in the supplementary figures (and note that the
qualitative results of all correlations are the same). For the
phylogenetic signal analyses, we likewise calculated Pagel's $\lambda$
(using \emph{phytools}\supercite{Revell2012phytools}) for all indices,
reporting all results in full in the supplementary materials (which
are again qualitatively identical to the results in the main text).

\printbibliography

\end{document}


%%% Local Variables:
%%% mode: latex
%%% TeX-master: t
%%% End:
